\renewcommand{\vec}[1]{\textbf{#1}}

\chapter{Methods}
\label{cha:methods}

In this thesis I try to address the concerns discussed in the introduction by training 
Probabilistic Circuits (PCs) through novel ways, specifically Score Matching (SM) and Sliced Score Matching (SSM).
Then I compare results from these methods with results from convential methods to train PCs.\\

\section{Creating a simple PC}
\label{sec:simple_pc}

Recall from section \ref{sec:pc} that the simplest variant a PC computes the weighted sum of arbitrary many input distributions. This is 
called a Mixture Model and the graph of an example with just two input distributions is shown in Figure \ref{fig:spn_gmm}. 

If the input distributions are Gaussian then it would be called a Gaussian Mixture Model and the expression to calculate the 
modelled density would be the following: 

\begin{equation}
    p(\vec x) =  \prod_{s=1}^S \sum_{k=1}^K \pi_k \mathcal{N}(\vec x_s|\boldsymbol{\mu}_k, \boldsymbol{\Sigma}_k) \\
    \label{eq:gmm}
\end{equation}

Where $S$ is the number of samples, $K$ is the number of components (distributions), $\pi_k$ the mixture weights and $\mathcal{N}(\vec x_s|\boldsymbol{\mu}_k, \boldsymbol{\Sigma}_k)$ 
a Gaussian component with mean $\boldsymbol{\mu}_k$ and covariance matrix $\boldsymbol{\Sigma}_k$.

However since we need to model the log-density $\log p(\vec x)$ for numerical stability we need to take the log of expression \ref{eq:gmm}: 

\begin{align}
    \log p(\vec x) & = \log \left(\prod_{s=1}^S \sum_{k=1}^K \pi_k \mathcal{N}(\vec x_s|\boldsymbol{\mu}_k, \boldsymbol{\Sigma}_k)\right) \notag \\
    & =  \sum_{s=1}^S \log \left(\sum_{k=1}^K \pi_k \mathcal{N}(\vec x_s|\boldsymbol{\mu}_k, \boldsymbol{\Sigma}_k)\right) 
    \label{eq:gmm_logp}
\end{align}

Here the problem arises that calculating $\sum_{k=1}^K \pi_k \mathcal{N}(\vec x_s|\boldsymbol{\mu}_k, \boldsymbol{\Sigma}_k)$ can 
produce very small values and when taking the logarithm this can lead to numerical instability but it can be mitigated with the LogSumExp trick.
First we take the exponential of the logarithm of the term inside the second sum. 

\begin{align}
    & \sum_{k=1}^K \exp\left( \log \left( \pi_k \mathcal{N}(\vec x_s|\boldsymbol{\mu}_k, \boldsymbol{\Sigma}_k) \right)\right) = \notag \\ 
    & \sum_{k=1}^K \exp\left( \log (\pi_k) + \log (\mathcal{N}(\vec x_s|\boldsymbol{\mu}_k, \boldsymbol{\Sigma}_k))\right)
    \label{eq:log_exp}
\end{align}

This is valid, because $\exp (\log(x)) = x$, since exponential and logarithm cancel each other out.
Plugging \ref{eq:log_exp} back into Equation \ref{eq:gmm_logp} yields the final numerically stable log density for our model. 

\begin{equation}
    \log p(\vec x) = \sum_{s=1}^S \log \sum_{k=1}^K \exp\left( \log \pi_k + \log \mathcal{N}(\vec x_s|\boldsymbol{\mu}_k, \boldsymbol{\Sigma}_k)\right)
    \label{eq:gmm_logp_final}
\end{equation}

I implemented this expression in python and used pytorch's torch.distributions.MultivariateNormal to compute the log density of a single 
gaussian component $\mathcal{N}(\vec x_s|\boldsymbol{\mu}_k, \boldsymbol{\Sigma}_k)$ and torch.logsumexp to calculate $\log \sum_{k=1}^K \exp()$.

Now to train this model on some data we need to formulate a objective function and the corresponding optimization problem. In the conventional way of Maximum Likelihood Estimation
(MLE) this is quite straight forward, since the objective function is exactly the density or in our case log density function 
$\log p(\vec x)$ and the optimisation problem is formulated as the following: 

\begin{equation}
    \max_{\substack{\boldsymbol \pi, \boldsymbol \mu, \boldsymbol \Sigma}} \log p(\vec x) = \sum_{s=1}^S \log \sum_{k=1}^K \exp\left( \log \pi_k + \log \mathcal{N}(\vec x_s;\boldsymbol{\mu}_k, \boldsymbol{\Sigma}_k)\right)
    \label{eq:gmm_objective}
\end{equation}

In plain text this means maximizing $\log p(\vec x)$ with respect to the weights $\boldsymbol \pi$, the means $\boldsymbol \mu$ and the covariance matrices $\boldsymbol \Sigma$.

I implemented the optimization of this objective with Expectation Maximization (EM) and Gradient Descent using pytorch's 
automatic differentiation capabilities. 
Note that for Gradient Descent to work we instead need to minimize the negative log likelihood $- \log p(\vec x)$, 
but this yields the same results as maximizing the log likelihood.

\section{Training via Score Matching}
\label{sec:gmm_score_matching}

To train the simple PC from above using Score Matching recall 
the objective function \ref{eq:sm_objective} from Section \ref{sec:sm}. 
With it we can formulate the optimization problem: 

\begin{equation}
    \max_{\substack{\boldsymbol \theta}} \mathcal{J}(\theta) = \mathbb{E}_{p_d(x)} \left[\text{tr} \left( \nabla_x^2 \log p_\theta(x) \right) + \frac{1}{2} \left\| \nabla_x \log p_\theta(x) \right\|^2 \right]
    \label{eq:sm_optimization_problem}
\end{equation}

Here $\log p_m(.;\theta)$ is of course the log density of our model as described in \ref{eq:gmm_logp_final}. \\
I used pytorch for the optimization with Gradient Descent. \\

The exact step-by-step way I calculated $\mathcal{J}(\theta)$ can be seen in Algorithm \ref{alg:sm}.

\begin{algorithm}
    \caption{Score Matching}
    \label{alg:sm}
    \begin{algorithmic}[1]  
        \Require $\log p_m(.;\theta), \vec x$
        \State $s_m(\vec x; \theta) \gets \nabla_x \log p_m(\vec x;\theta), \vec x$
        \State $\mathcal{J} \gets x$
        \State \Return $\mathcal{J}$
    \end{algorithmic}
\end{algorithm}


To calculate the gradients as in $\text{grad}(\log p_m(\vec x;\theta), \vec x)$ I used torch.autograd.grad. 


\section{Training via Sliced Score Matching}

Similar to Section \ref{sec:gmm_score_matching} we take the objective function \ref{eq:ssm_vr} from Section \ref{sec:ssm} and formulate
the optimization problem:

\begin{equation}
    \min_{\substack{\boldsymbol \theta}}  \mathcal{J}(\theta; p_\vec{v}) = \mathbb{E}_{p_v} \mathbb{E}_{p_d} \left[ \vec{v}^T \nabla_\vec{x} s_\theta(\vec{x}) \vec{v} + \frac{1}{2} \norm{ s_\theta(\vec{x}) }^2_2 \right]
    \label{eq:ssm_optimization_problem}
\end{equation}

I again used pytorch for optimization with Gradient Descent and the exact computation of $\mathcal{J}(\theta; p_\vec{v})$ can be seen 
in Algorithm \ref{alg:ssm}.

\begin{algorithm}
    \caption{Sliced Score Matching}
    \label{alg:ssm}
    \begin{algorithmic}[1]  
        \Require $\log p_m(.;\theta), \vec x$
        \State $s_m(\vec x; \theta) \gets \nabla_x \log p_m(\vec x;\theta), \vec x$
        \State $\mathcal{J} \gets x$
        \State \Return $\mathcal{J}$
    \end{algorithmic}
\end{algorithm}

\section{Using more complex PCs}

While the model proposed in Section \ref{sec:simple_pc} works well for relatively simple tasks like 2 dimensional density estimation 
with not too complex distributions, it doesn't perform very well with complex higher dimensional data like images. 

So to get a wider variety of results I decided to also use a state-of-the-art framework for Probabilistic Circuits called 
EinsumNetworks \cite{einsum} for modelling more complex higher dimensional datasets, especially images. 

For more details on how EinsumNetworks work please refer to the \cite{einsum} but in principle 
