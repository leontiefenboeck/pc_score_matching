\chapter{Discussion}
\label{cha:discussion}

\section{Interpretation of Experimental Results}

Looking at the best possible results \ref{sec:best_results} from the 2D Density Estimation Experiments, we can see that
Maximum Likelihood Estimation with Gradient Descent provides the highest Log-Likelihood values for all scenarios. 
This could be somewhat expected, since here we are directly maximizing the Likelihood, whereas with Score Matching the Likelihood
is only indirectly optimized by minimizing the Score Matching Objective which comes from Fisher Divergence. 
However, also when looking at the density and sample plots, Gradient Descent objectively produces the nicest results, meaning the least 
noisy and the most accurately distributed, so for example no made up high density regions where there shouldn't be one. 
This wasn't necessarily expected.  
Also not expected was that where EM provided similar results to Gradient Descent for small $K$, EM struggled with large $K$, where it found
really small very high density Gaussians that messed up the sample plots. 

Still considering these plots it was interesting how well Sliced Score Matching with only $1$ slice worked as an approximation for Exact Score Matching,
which we empirically confirmed later. 

When looking at the Performance Analysis \ref{sec:2d_exp2} the main takeaways are that EM optimizes the fastest epoch-wise and overall time-wise, 
though GD is only slightly slower. As expected the many additional calculations for the Score Matching Algorithms provide a huge 
performance penalty. 

For random initialization \ref{sec:2d_exp3} it was somewhat surprising how similar all algorithms performed when compared 
to their deterministic KMeans counterpart. The Log-Likelihood
values were usually only slightly worse and the same goes for the density and sample plots. However we noticed that for the more 
complex spirals dataset the difference was already larger than for the simple halfmoon dataset. So we expect that the more complex 
the dataset is, the worse random initialization performs.  

Overall we can conclude that for 2D Density Estimation, Gradient Descent provided the best results, had the best robustness to 
different hyperparameter settings and was nearly tied as the fastest algorithm to optimize. 