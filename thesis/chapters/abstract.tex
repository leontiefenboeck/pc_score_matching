\chapter{Abstract}

\emph{Side-Channel Attacks} (SCAs) have demonstrated remarkable potential to attack cryptographic systems and compromise their security by exploiting unintentional information leakage through various physical channels. Using different methods, we can encode the information present in the leakage as probability distributions over intermediate values (input distributions). State-of-the-art methods like the \emph{Soft Analytical Side-Channel Attack} (SASCA) combine these input distributions with knowledge about the cryptographic algorithm by representing the system under attack as a factor graph, in which an attacker performs probabilistic inference to reason about likely key hypotheses. In many real-world scenarios, SASCA must resort to loopy belief propagation, an approximate inference algorithm that does not guarantee convergence or accurate estimates.
In this work, we study the applicability of a class of tractable probabilistic models, namely \emph{Probabilistic Circuits} (PCs), to SCA inference problems. Using techniques from \emph{Knowledge Compilation}, we find that we can tractably represent parts of the \emph{Advanced Encryption Standard} (AES) algorithm as a PC and use this fact to replace loopy parts of a factor graph with a PC. Our experiments show that, under some assumptions, we can utilize PCs to tractably perform \emph{exact} inference using a sparse approximation of input distributions, which significantly outperforms the SASCA in a real-world attack scenario. We also provide an information-theoretic upper bound on the approximation error of the input distributions.
Moreover, we experiment with \emph{Deep Learning} based SCAs and show the close relationship between our proposed inference method and recent developments in the field of \emph{neuro-symbolic learning}. In this context, we show that we can also learn input distributions that are amenable to exact inference without the need for sparse approximations, albeit at the cost of expressiveness.