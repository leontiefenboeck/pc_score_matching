\chapter{Conclusions and Future Work}
\label{cha:conclusions}

In this thesis we experimentally tested different methods to train Probabilistic Circuits (PCs), specifically
we wanted to examine how recent developments in training Energy or Score based Models, namely 
Sliced Score Matching would transfer to PCs. 
We achieved this by implementing the algorithms and conducting experiments on 2D Density Estimation
and Image Modelling. In all relevant experiments we compared performance by producing samples 
(and density plots) and measuring the Log-Likelihood.

Concluding we can say, we did not recognize any obvious benefits to use Score Matching, wether it being Sliced 
or Exact, over the conventional Maximum Likelihood Estimation (MLE) using either Gradient Descent 
or Expectation Maximization, when just looking at the Log-Likelihood measurements or the time it 
took to train a model. 
When considering the samples, especially for the Image Modelling task, we saw that Sliced Score Matching
produced very sharp and less noisy samples compared to the MLE counterparts.
These huge improvements in image quality, however came with the downside that 
samples were less diverse in the single-class scenario and in the multi-class scenario
the different classes were not even recognizable.

For future work we suggest to further investigate the reasons for the increased sharpness and the 
lack of diversity as well as the superposition of classes in SSM, with the goal of finding 
an algorithm that combines diversity and sample quality. 
For now to provide a possible solution, at least when trying to train on an entire dataset with 
multiple classes, one could use Sliced Score Matching to train a model on each class individually and 
then create a mixture to combine them. Though this does not solve the problem in essence.

Another path of possible future research is to use exact Score Matching in combination with
PCs. We hypothesize that, due to the nature of PCs, exact Score Matching 
could be implemented in a more efficient way than in general Energy Based Models, reducing 
the computational overhead and making it more feasible to use in practice.