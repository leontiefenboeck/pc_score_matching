\chapter{Kurzfassung}

\emph{Seitenkanalangriffe} (SCAs) haben bemerkenswertes Potenzial gezeigt, kryptografische Systeme anzugreifen und ihre Sicherheit durch Ausnutzung unbeabsichtigter Informationslecks über verschiedene physische Kanäle zu kompromittieren. Mit verschiedenen Methoden können wir die in dem Leck vorhandene Information als Wahrscheinlichkeitsverteilungen über Zwischenwerte (Eingangsverteilungen) kodieren. Modernste Methoden wie \emph{Soft Analytical Side-Channel Attacks} (SASCAs) kombinieren diese Eingangsverteilungen mit Wissen über den kryptografischen Algorithmus, indem sie das angegriffene System als Faktorgraph darstellen, in dem ein Angreifer probabilistische Inferenz durchführt, um wahrscheinliche Schlüsselhypothesen zu generieren. In vielen realen Szenarien muss SASCA auf \emph{Loopy Belief Propagation} zurückgreifen, einem ungefähren Inferenzalgorithmus, der weder Konvergenz noch genaue Schätzungen garantiert. In dieser Arbeit untersuchen wir die Anwendbarkeit einer Klasse von probabilistischen Modellen, den sogenannten \emph{Probabilistic Circuits} (PCs), für Inferenzprobleme in SCAs. Wir stellen fest, dass wir mithilfe von \emph{Knowledge Compilern} Teile des \emph{Advanced Encryption Standard} (AES) Algorithmus als PC darstellen können und verwenden diese Tatsache, um zyklische Teile eines Faktorgraphen durch einen PC zu ersetzen. Unsere Experimente zeigen, dass wir unter bestimmten Annahmen PCs verwenden können, um \emph{exakte} Inferenz mit einer dünnbesetzten Approximation der Eingabeverteilungen durchzuführen, was SASCA in einem realen Angriffsszenario deutlich übertrifft. Wir zeigen au{\ss}erdem eine informationstheoretische Obergrenze für den Annäherungsfehler der Eingabeverteilungen. Darüber hinaus experimentieren wir mit Deep Learning-basierten SCAs und zeigen die enge Beziehung zwischen unserer vorgeschlagenen Inferenzmethode und jüngsten Entwicklungen im Bereich des \emph{neurosymbolischen} Lernens.
In diesem Zusammenhang zeigen wir, dass wir auch Eingabevertellungen erlernen können, die für exakte Inferenz geeignet sind, ohne dass  Annäherungen der Verteilungen erforderlich sind --- wenn auch auf Kosten der Ausdrucksfähigkeit dieser Verteilungen.